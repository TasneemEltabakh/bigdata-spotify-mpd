\documentclass[12pt,a4paper]{article}

% Packages
\usepackage[utf8]{inputenc}
\usepackage[T1]{fontenc}
\usepackage{graphicx}
\usepackage{hyperref}
\usepackage{amsmath}
\usepackage{booktabs}
\usepackage{listings}
\usepackage{xcolor}
\usepackage{geometry}
\usepackage{float}

% Page geometry
\geometry{margin=1in}

% Code listing style
\lstset{
    language=Python,
    basicstyle=\ttfamily\small,
    keywordstyle=\color{blue},
    commentstyle=\color{gray},
    stringstyle=\color{orange},
    showstringspaces=false,
    breaklines=true,
    frame=single
}

% Title information
\title{Big Data Analytics Project:\\Spotify Million Playlist Dataset Analysis}
\author{[Author Name]} % TODO: Replace with actual author name(s)
\date{\today}

\begin{document}

\maketitle

\begin{abstract}
This report presents a comprehensive big data analytics project utilizing the Spotify Million Playlist Dataset (MPD). Using Databricks and PySpark, we implement data storage, cleaning, exploratory data analysis, and ETL pipelines to derive insights from playlist data. The project demonstrates scalable data processing techniques applicable to real-world big data challenges.
\end{abstract}

\tableofcontents
\newpage

% =============================================================================
\section{Introduction}
% =============================================================================

\subsection{Project Overview}
% TODO: Provide an overview of the project objectives and scope

\subsection{Dataset Description}
The Spotify Million Playlist Dataset (MPD) contains:
\begin{itemize}
    \item 1 million user-generated playlists
    \item Over 66 million unique tracks
    \item Playlist metadata including name, number of tracks, and followers
    \item Track information including artist, album, and duration
\end{itemize}

\subsection{Technologies Used}
\begin{itemize}
    \item \textbf{Databricks}: Cloud-based big data platform
    \item \textbf{Apache Spark}: Distributed computing framework
    \item \textbf{PySpark}: Python API for Spark
    \item \textbf{Delta Lake}: Open-source storage layer
\end{itemize}

% =============================================================================
\section{Data Storage}
% =============================================================================

\subsection{Data Ingestion}
% TODO: Describe the data ingestion process

\subsection{Storage Architecture}
% TODO: Describe the Delta Lake storage architecture

\subsection{Data Schema}
% TODO: Document the data schema

% =============================================================================
\section{Data Cleaning and Preparation}
% =============================================================================

\subsection{Data Quality Assessment}
% TODO: Describe data quality issues identified

\subsection{Cleaning Procedures}
% TODO: Document cleaning steps performed

\subsection{Data Validation}
% TODO: Describe validation procedures

% =============================================================================
\section{Exploratory Data Analysis}
% =============================================================================

\subsection{Playlist Analysis}
% TODO: Present playlist statistics and visualizations

\subsection{Track Analysis}
% TODO: Present track statistics and visualizations

\subsection{Artist Analysis}
% TODO: Present artist statistics and visualizations

\subsection{Key Findings}
% TODO: Summarize key insights from EDA

% =============================================================================
\section{ETL Pipeline}
% =============================================================================

\subsection{Pipeline Architecture}
% TODO: Describe the ETL pipeline design

\subsection{Extract Phase}
% TODO: Document data extraction process

\subsection{Transform Phase}
% TODO: Document data transformation logic

\subsection{Load Phase}
% TODO: Document data loading process

\subsection{Data Warehouse Schema}
% TODO: Present the star/snowflake schema design

% =============================================================================
\section{Results and Discussion}
% =============================================================================

\subsection{Performance Metrics}
% TODO: Present performance benchmarks

\subsection{Insights and Findings}
% TODO: Discuss key insights derived from the analysis

\subsection{Challenges and Solutions}
% TODO: Document challenges faced and solutions implemented

% =============================================================================
\section{Conclusion}
% =============================================================================

% TODO: Summarize project outcomes and future work

% =============================================================================
\section{References}
% =============================================================================

\begin{thebibliography}{9}

\bibitem{spotify_mpd}
Spotify Research. (2018). 
\textit{The Million Playlist Dataset}. 
\url{https://www.aicrowd.com/challenges/spotify-million-playlist-dataset-challenge}

\bibitem{databricks}
Databricks Documentation.
\url{https://docs.databricks.com/}

\bibitem{spark}
Apache Spark Documentation.
\url{https://spark.apache.org/docs/latest/}

\bibitem{delta_lake}
Delta Lake Documentation.
\url{https://docs.delta.io/latest/index.html}

\end{thebibliography}

% =============================================================================
\appendix
\section{Appendix: Code Snippets}
% =============================================================================

% TODO: Include relevant code snippets

\end{document}
